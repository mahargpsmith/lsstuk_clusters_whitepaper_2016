\documentclass[a4paper,11pt]{article}
\usepackage{graphicx,url}
\usepackage{draftwatermark}
\SetWatermarkText{DRAFT}
\SetWatermarkScale{8}
%\usepackage[big,compact,sf]{titlesec}
%\usepackage{sidecap}
\usepackage{graphics,graphicx}
%\usepackage{verbatim}
%\usepackage{wrapfig}
%\usepackage{floatflt}
%\usepackage{times}
%\usepackage{multicol}
\usepackage{color}
\newcommand{\red}{\textcolor{red}}

%\usepackage{sober}
%\usepackage[small,compact,sf]{titlesec}

\newcounter{fred}

\newcommand{\mic}{\mu{\rm m}}
\def\es{\mathrel{\rm e^-/s}}
\def\pix{\mathrel{\rm pix}}
\def\elec{\mathrel{\rm e^-}}
\def\kpc{\mathrel{\rm kpc}}
\def\Gpc{\mathrel{\rm Gpc}}
\def\Msol{\mathrel{\rm M_{\odot}}}
\def\fsub{\mathrel{f_{\rm sub}}}
\def\Mtot{\mathrel{M_{\rm tot}}}
\def\ls{\mathrel{\hbox{\rlap{\hbox{\lower4pt\hbox{$\sim$}}}\hbox{$<$}}}}
\def\gs{\mathrel{\hbox{\rlap{\hbox{\lower4pt\hbox{$\sim$}}}\hbox{$>$}}}}
\def\Msolpyr{\mathrel{\rm M_{\odot}\,yr^{-1}}}
\def\mas{\mathrel{\rm mas}}
\def\pc{\mathrel{\rm pc}}
\def\Ho{\mathrel{H_{\rm 0}}}
\def\oM{\mathrel{\Omega_{\rm M}}}
\def\oL{\mathrel{\Omega_{\rm \Lambda}}}
\def\kms{\mathrel{\rm km\,s^{-1}}}
\def\ms{\mathrel{\rm m\,s^{-1}}}
\def\m{\mathrel{\rm m}}
\def\nm{\mathrel{\rm nm}}
\def\mm{\mathrel{\rm mm}}
\def\cm{\mathrel{\rm cm}}
\def\km{\mathrel{\rm km}}
\def\um{\mathrel{\mu{\rm m}}}
\def\ang{\mathrel{\rm \AA}}
\def\Mpc{\mathrel{\rm Mpc}}
\def\ksec{\mathrel{{\rm ksec}}}
\def\mag{\mathrel{\rm mag}}
\def\Gyr{\mathrel{\rm Gyr}}
\def\Hz{\mathrel{\rm Hz}}
\def\MHz{\mathrel{\rm MHz}}
\def\GHz{\mathrel{\rm GHz}}
\def\THz{\mathrel{\rm THz}}
\def\PHz{\mathrel{\rm PHz}}
\def\EHz{\mathrel{\rm EHz}}
\def\Js{\mathrel{\rm Js}}
\def\J{\mathrel{\rm J}}
\def\W{\mathrel{\rm W}}
\def\eVs{\mathrel{\rm eV\,s}}
\def\eV{\mathrel{\rm eV}}
\def\K{\mathrel{\rm K}}
\def\Jy{\mathrel{\rm Jy}}
\def\mJy{\mathrel{\rm mJy}}
\def\uJy{\mathrel{\rm \mu Jy}}
\def\sr{\mathrel{\rm sr}}
\def\rad{\mathrel{\rm rad}}
\def\deg{\mathrel{\rm deg}}
\def\degsq{\mathrel{\rm deg}^2}
\def\fwhm{\mathrel{\rm FWHM}}
\def\fried{\mathrel{r_0}}
\def\fo{\mathrel{f_{\rm o}}}
\def\fe{\mathrel{f_{\rm e}}}
\def\s{\mathrel{\rm s}}
\def\dol{\mathrel{D_{\rm OL}}}
\def\dos{\mathrel{D_{\rm OS}}}
\def\dls{\mathrel{D_{\rm LS}}}


\newcommand{\captionfonts}{\small}
\makeatletter  % Allow the use of @ in command names
\long\def\@makecaption#1#2{%
  \vskip\abovecaptionskip
  \sbox\@tempboxa{{\captionfonts #1: #2}}%
  \ifdim \wd\@tempboxa >\hsize
    {\captionfonts #1: #2\par}
  \else
    \hbox to\hsize{\hfil\box\@tempboxa\hfil}%
  \fi
  \vskip\belowcaptionskip}
\makeatother   % Cancel the effect of \makeatletter


\setlength{\textwidth}{172mm} 
\setlength{\textheight}{260mm}
\setlength{\topmargin}{-20mm} 
\setlength{\oddsidemargin}{-5mm}
\setlength{\evensidemargin}{10mm} 
\setlength{\headheight}{5mm}
\setlength{\headsep}{5mm} 
\setlength{\hoffset}{0in}
\setlength{\voffset}{0in}

\parskip=2truemm                       % Paragraph spacing
\parindent=4truemm                       % Paragraph indentation

\begin{document}

\pagestyle{myheadings}\markright{LSST:UK Galaxy Clusters White Paper 2016}

\sloppy

\pagestyle{empty}

~\vspace{70mm}

\centerline{\LARGE\bf LSST:UK Galaxy Clusters}
\bigskip\bigskip\bigskip
\centerline{\Large\bf High Level Science Interests and Science Requirements}
\medskip
\centerline{\Large\bf of the UK Galaxy Clusters Community}
\medskip
\centerline{\Large\bf as they relate to LSST}
\bigskip\bigskip\bigskip
\centerline{\Large\bf White Paper 2016}

\vspace{90mm}

\large
\noindent{\bf Last updated}: June 16, 2016

\noindent{\bf Contributors}: Graham P.\ Smith, et al. {\it [add your names here]}


\newpage
\pagestyle{myheadings}
\setlength{\topmargin}{-10mm}
\setlength{\textheight}{255mm}

\tableofcontents

\newpage

\section{Introduction}

{\it [Responsible: Graham Smith]}

\noindent
The purpose of this document is to summarise and communicate the UK
galaxy cluster community's interests as they relate to future
exploitation of LSST.  Section 2 describes our interests and the high
level requirements that they place on LSST data and combining data
from LSST with data from other upcoming facilities, including
\emph{Euclid} and \emph{e-ROSITA}.  Section 3 distils the common
themes from our varied interests to form the basis for discussion with
other communities within LSST (both UK-based, and internationally),
and colleagues leading the development of other facilities.

It is also envisaged that this document, and the discussions that it
stimulates, will help to shape requests for funding for galaxy cluster
science in the future.  The LSST:UK Phase B proposal to STFC's PPRP
will be a key opportunity to secure funding to support the development
of computer algorithms that the UK community will need in order to
delivering world-class galaxy cluster science with LSST.

The initial names attached to science interests in Section 2 is based
on participation/discussion at the inaugural LSST:UK Clusters meeting
on June 7, 2016.  However this document is open for anyone in the UK
to join and contribute to.  The aim is to be succinct, so please try
to stick to no more than one page per science interest in Section 2.

We will need to add a section that discusses how our interests,
requirements on the data, and strengths compare with those of the US
community.

\section{Science Interests}

UK astronomers play leading international roles in numerous aspects of
galaxy cluster research that are relevant to the future exploitation
of LSST data.  

Recent highlights include constraints on the nature of dark matter
(Harvey et al.\ 2015, Sci, 347, 1462), measurements of intracluster
light at $z=1$ (Burke et al., 2012, MNRAS, 425, 2058), systematic
census of acticity in brightest cluster galaxies (Green et al.,
arXiv:1606.01251), cosmological hydrodynamical simulations that
reproduce the observed properties of galaxy clusters and groups
(McCarthy et al., 2016), testing chameleon gravity with clusters from
the XMM Cluster Survey (Wilcox et al., 2015, MNRAS, 452, 1171),
exploring the structure of cluster cores using Hubble Frontier Fields
observations (Jauzac et al., 2015, MNRAS, 452, 1437; Jauzac et
al.\ arXiv:1606.04527), galaxy evolution in clusters and groups (Sean
McGee, others), discovery of protoclusters around high-z radio
galaxies (Hatch et al., ...), X-ray scaling relations of clusters
(Giles et al.\ 2016, A\&A, 592, 3), weak-lensing calibration of the
masses of galaxy groups and clusters (Smith et al., 2016, MNRAS, 456,
L74; Okabe \& Smith, arXiv:1507.04493; Lieu et al.\ 2016, A\&A, 592,
4).

\emph{Explicitly mention surveys that we have leading roles in?  GEEC,
  GEEC2, XXL, CARLA, XCS, DES, GOGREEN, LoCuSS, ...}

The following sections outline the science interests of the UK's
galaxy cluster community as they relate to LSST, highlighting the
requirements that these interests place on LSST data.

\subsection{Intracluster Light and low surface brightness emission}

{\it [Responsible: Chris Collins]}

\noindent
Understanding the build up of stellar mass at the centres of galaxy
clusters is a key component to our understanding of galaxy
evolution. The centres of rich clusters are dominated by Brightest
Cluster Galaxies (BCG) which are the most luminous galaxies in the
universe emitting photospheric light. Observational studies have
consistently indicated that the masses of BCGs change by a factor no
larger than $30\%$ over the cosmic time out to a redshift $ z~1$, with
some results suggesting considerably less change (e.g. Collins et al.,
2009, Nature, 458, 603; Zhang, et al., 2016, ApJ, 816, 98). These
results are in themselves interesting as they challenge current model
predictions, particularly compared to semi-analytic simulations of
galaxy evolution. Furthermore model independent estimates of the
expected growth rates of BCGs from counts of satellite galaxies in the
cores of clusters, indicate that significant stellar mass is available
to grow the BCG based on dynamical friction timescale arguments. The
stellar mass in fact manifests itself not as BCG growth but as diffuse
and extended low surface brightness Intra Cluster Light (ICL) which
appears to grow rapidly since a redshift $z~1$ (Burke et al., 2015,
MNRAS, 449, 2353). Furthermore, the component ICL in nearby rich
clusters is large, and can it appears can even dominate the BCG
stellar mass. However there is no consensus on how the ICL grows with
cosmic time or even how the ICL fraction depends on cluster mass or
other parameters, with different conclusions in the literature largely
due to varying data quality.

Our interest in using LSST data is to quantify the ICL as a function
of cluster properties and redshift to at least $z~1.5$. The faint SB
limits of LSST stacked data will provide powerful comparisons with the
new generation high resolution galaxy simulations such as EAGLES which
can make predictions at LSST depths. In order to achieve this it is
important to optimise the sky background, flat fielding and PSF
correction techniques on large angular scales for low surface
brightness emission. Here it is worth noting that LSST will detect the
ICL in nearby clusters in the early exposures, involving detections
out to at least 1 degree in the case of Virgo. Existing survey data do
not provide sufficiently robust optimisation at low surface
brightness. Recent work using KiDSS (Kelvin et al., 2016, in
preparation) indicates that the ICL cannot satisfactorily be measured
using the current survey data release due to sky background estimation
techniques optimised for galaxies. Scattered light also adds a
systematic component to the measured extended intensities. Our own
work on deep VLT data with HAWK-I on the VLT in the near-IR J window
indicates that consistently deep surface brightness levels (defined as
recovering $80\%$ of the total flux and estimated using fake ICL
signals) can be obtained using a time averaged running median filter
to estimate sky and an appropriate random dither pattern. Similar
points concerning the PSF have also recently been made (Lombilla et
al., 2016, LSST Belgrade) concerning LSST measurements of the low
surface brightness thick disk components of edge-on galaxies. There is
therefore an urgent need to simulate the effects of LSST observing
techniques and the determination of the PSF on large angular scales to
extract the maximum information on the faint cosmic light from
ultra-deep LSST imaging at depths reaching $r \simeq 30-33$ per
arcsec$^{-2}$.

\subsection{High redshift clusters and protoclusters}

{\it [Responsible: Nina Hatch and Malcolm Bremer]}

\noindent We will use the LSST survey and deep drilling fields to
search for distant ($z >1$) clusters and protoclusters to study the
formation of clusters and their member galaxies. Cluster and
protocluster detection algorithms will concentrate on searching for
galaxy overdensities in redshift space which exhibit strong
Balmer/4000\AA break features over large areas ($\sim$
10\,arcmin$^{-2}$). Based on recent studies (e.g. Chaing et al. 2014),
(proto)cluster detection will require photometric redshifts with at
least $\Delta z/ (1+z)\sim2.5$\% precision at $z>1$. This requires a
strong synergy between Euclid and LSST data as the Euclid Y, J and H
images are essential to span over the Balmer/4000\AA\ break for
$z>1.5$ galaxies.

The depth of the final LSST catalogues are sufficient to detect
significant numbers of red galaxies within each (proto)cluster at
redshifts up to $z\sim1.7$. Clusters and protoclusters up to
$z\sim2.5$ can be detected in the deep drilling fields using similar
algorithms, and these fields will produce cleaner cluster samples at
$z<1.7$ due to their lower photometric redshift uncertainty.

For protoclusters at $z>2.5$, the Balmer/$4000$\AA\ break shifts
beyond the wavelength range of LSST and Euclid. Furthermore, the red
sequences of such distant protoclusters are less prominent (perhaps
non-existent). We therefore must use the Lyman break feature of
galaxies to locate $z>2.5$ protoclusters.

To locate protoclusters the images must be deep enough to robustly
identify a galaxy overdensity across a few sq. arcmin. This requires
us to detect at least $\sim2$ LBGs per sq. arcmin. To achieve this the
filter redward of the Lyman break must reach depths of 26-27 mag (AB),
and the filter blueward of the break must reach 27-28 mag [but more
  detailed simulations are needed]. This means:\\
\noindent $\bullet$~$z\sim3$ U-band dropouts will only be detected to
sufficient depths in the deep drilling fields as the U-band in the
main field survey is too shallow. \\
\noindent $\bullet$ ~Dropouts in the g- and r-bands ($3.5<z<4.5$) will
be detected to sufficient depths to locate protoclusters in the main
survey. \\
\noindent $\bullet$ ~The surface density of i- and z-band dropouts
will be too low to robustly identify protoclusters in both main and
deep drilling fields. [Although this is subject to change as the
  depths of the deep fields are not certain at this stage]

In summary, combining the LSST and Euclid surveys allows us to search
for $1<z<1.7$ clusters and $3.5<z<4.5$ protoclusters across the main
survey using photometric redshifts and Lyman break dropout
techniques. The deep drilling fields allow us to search for $z>1.7$
(proto)clusters using photometric redshifts, and $z\sim3$
protoclusters using U-band dropouts. The main requirement for locating
distant clusters and protoclusters is a good synergy between Euclid
and LSST data products. We require accurate photometric catalogues
based on Euclid detection images. The deep drilling fields are very
important and infrared (3.6 and 4.5\,$\mu$m) coverage of these fields
is highly desirable.

In preparation for LSST data we should develop and test algorithms to
locate distant clusters and protoclusters. Realistic simulated light
cones would help test these algorithms. Such light comes need to be
sufficiently large to contain significant numbers of high redshift
clusters and protoclusters, and they would be most useful if the
luminosity function of galaxies (at LSST and Euclid wavelengths)
matched that of the local Universe and up to $z\sim4$.

\subsection{Numerical simulations}

\subsection{Galaxy cluster mass calibration}

{\it [Responsible: Graham Smith]}

\noindent
Accurate absolute mass calibration of galaxy cluster samples out to
$z\simeq1$ and down to $M_{200}\sim5\times10^{13}M_\odot$ is essential
if clusters are to deliver a competitive constraint on dark energy.
This mass calibration will come from weak-lensing measurements using
optical survey data, because the basic measurements (galaxy shapes)
will be available directly from the optical survey data, and
weak-lensing probes directly the total mass distribution of clusters.
Recent progress on the mass calibration of clusters includes
measurements of massive clusters at $z<0.3$ with the lowest systematic
biases to date (Okabe \& Smith, 2016), and applying weak-lensing
techniques to poor clusters and rich groups (Lieu et al.\ 2016).

The most difficult systematic biases to control are those relating to
the selection of background galaxies, and then placing the background
galaxies accurately along the line of sight behind the clusters.  In
the LSST era the background galaxies will have apparent magnitudes of
${\rm AB}\sim23-27$, i.e.\ dominated by galaxies beyond the current
spectroscopic limit.  To obtain photometric redshifts with well
understood uncertainties, and to prove that the uncertainties are well
understood therefore requires deep spectroscopy way beyond the current
spectroscopic limit of ${\rm AB}\sim23-24$.  This is true even if we
accept that a useful level of spectroscopic completeness at the very
faintest limits is simply unachievable.

The challenge of controlling biases due to photometric selection and
photometric redshift estimation of background galaxies will be most
severe for high redshift clusters.  This is because the same optical
survey data of a given sensitivity will be used to select and
characterize the background galaxies for clusters independent of the
cluster redshifts.  Therefore higher redshift clusters are closer to
the background galaxies than the lower redshift clusters.
Consequently, the lensing kernel ($D_{\rm LS}/D_{\rm S}$; with which
the lensing signal scales linearly) is always a steeper function of
background galaxy redshift for higher redshift clusters than for lower
redshift clusters (Figure~1).

\noindent
\begin{minipage}{60mm}
%  \begin{figure}
    \includegraphics[width=\hsize]{dlsds.png}
\end{minipage}
\hspace{5mm}
\begin{minipage}{105mm}
  Figure~1: Dls/Ds ratio as a function of background galaxy redshift
  for cluster lenses at $z=0.2,0.6,1.0$. At any given background
  galaxy redshift the curve for the higher redshift clusters is
  steeper than the curve for lower redshift clusters.  Therefore the
  potential systematic bias due to mis-estimation of photometric
  redshifts is larger for higher redshift clusters.
\end{minipage}
%\end{figure*}

Building on the recent demonstration by Okabe \& Smith (2016) that
galaxies redder than the red sequence of cluster members in the
$(V-i)/i$ colour-magnitude plane offers a safe and calibratable method
to select galaxies behind massive galaxy clusters at $z\simeq0.2$, it
is important to explore how these methods can be extended to higher
redshifts.  Assuming for now that the rest-frame equivalents of $V$
and $i$-bands are the optimal filters for a red galaxy selection,
would motivate consideration of the $(i-J)$ colours for selection of
galaxies behind clusters at $z\simeq1$.  It is therefore clear that
photometry from \emph{Euclid} has an important role to play in
LSST-based cluster weak-lensing studies.  This point is further
amplified by the fact that merged LSST and \emph{Euclid} photometric
catalogues will be important for a broad range of extragalactic
science.  Given that this merged catalogue will be used to select
weakly-lensed background galaxies for which accurate shape
measurements are available, it is essential that the joint
LSST/\emph{Euclid} photometric catalogue production is based on the
same source identification as that used for the galaxy shape
measurement.  This begs the question of whether the source
identification for merged LSST/\emph{Euclid} photometric catalogues
should come from LSST or \emph{Euclid}?

Strategies that can be explored to improve/verify the accuracy of
photometric redshift estimates for galaxy cluster science include
choice of number, depth, and sky location of the deep drilling fields,
ultra-deep spectroscopic surveys of the deep drilling fields,
exploring how the \emph{Euclid} grism spectroscopy can assist with
spectroscopic calibration of photometric redshifts, development of
algorithms that include optical/X-ray/SZ information about the
over-densities along the line of sight through candidate clusters.

Galaxy clusters represent crowded lines of sight through the universe.
As such the requirements on the accuracy of deblended photometry are
more demanding than for typical lines of sight.  It is therefore
important to test the deblending capabilities of the LSST Data
Management team's software stack (DM Stack) on known clusters down to
$\sim5\times10^{13}M_\odot$ and out to $z\simeq1$.  It would be
sensible to do these tests on the most LSST-like data available today,
for example data obtained with Subaru telescope on massive low
redshift clusters, and also Hyper-Suprime-CAM (HSC) survey data.  On
this latter point, HSC has observed the XXL-N field to full survey
depth already.  This field contains clusters across the required
redshift and mass range.  The UK is well-placed to lead in this area
given the leading roles that we play in LoCuSS and XXL.  

\subsection{Machine learning and cluster detection methods}

{\it [Responsible: Jim Geach]}

\subsection{Brightest Cluster Galaxies}

{\it [Responsible: Alastair Edge]}

\noindent
The Brightest Cluster Galaxy (BCG) in a cluster is in most
clusters substantially brighter than all other cluster
members (by 1--1.5~mag) and found at the dynamical centre
of the cluster at, or close to, the X-ray peak (Sanderson
et al. 2009). These galaxies represent the most massive 
stellar systems known and their properties can be used
to constrain their formation history and how it relates
to the evolution of their host cluster.

Most BCGs are dominated by an old stellar population
and share the same colour-magnitude sequence with the
other early type galaxies in the cluster. However, there
is a significant population of BCGs that exhibit recent
star formation and/or an AGN core (Crawford et al. 1999).
This subset of ``active'' BCGs are predominately found in
the most X-ray luminous clusters (Green et al 2016) and
have associated optical line emission and cold molecular
gas (Edge 2001). The properties of the intracluster gas 
surrounding the BCG appear to be the dominant factor
in triggering this activity (Cavagnolo et al 2008) with
a sharp threshold in central entropy marking the 
point where $>$90\% of BCGs below show optical line emission 
and $<$5\% of those above have lines. 

Therefore, if it is possible to identify which BCGs are
significantly bluer expected from the observed colours
of the cluster population, we can very efficiently
photometrically select these objects and compare
their radio, MIR and X-ray properties to the 
more passive BCGs. In the radio the average power
of a BCG which exhibits optical line emission is
an order of magnitude higher than that in non-line
emitting systems (Hogan et al. 2015). 

The depth and coverage of LSST, when combined
with NIR and MIR data from Vista VHS and AllWISE,
will be able to select these peculiar BCGs in
any cluster within $z\sim 1.0$ and the more
extreme examples out to $z>1.5$. The combination
of multi-band photometry and multi-frequency
radio data will allow us to determine if the 
AGN and star-formation activity in BCGs evolves
significantly over the majority of the lifetimes
of all clusters.

In addition to the photometric properties
of BCGs, LSST will also provide important
morphological information. For instance,
N-body simulations suggest that the majority
of the growth of BCGs is through the merger
with other massive, early type galaxies 
(De Lucia \& Bliazot 2007). Therefore a
significant fraction of BCGs should have 
the multiple nuclei expected for two 
galaxies in the latter stages of these
massive mergers. These systems are known
but a significant fraction may be chance
alignments given the very high surface density of galaxies
in a cluster core. The quality and coverage of
LSST will provide the statistics to assess
the probabilty of chance alignment and the
relative brightness of the components
in a BCG to the likely merger rate of BCGs
can be determined as a function of redshift
and cluster mass.

The LSST imaging will also identify BCGs
with extended stellar envelopes,
usually refered to as cD haloes,
that extend on $\approx$100~kpc scales
blending into what can be defined
as Intracluster Light (see Chris's section?).
This extended halo is often quite asymmetric
and can contain shells and other evidence
of merger activity. Again, the depth and
consistency of the LSST data will allow us
to determine the properties of these 
extended haloes with other cluster properties
over a wide range in redshift.

Finally, a small fraction of BCGs contain a
sufficiently bright active nucleus that
variability can be detected. The most prominent
example of this is NGC1275 in the Perseus cluster
(Kingman \& O'Connell et al. 1979) which has shown 
two episodes of strong AGN activity in the past
50 years (Dutson et al. 2014). The optical
variability is significant (0.2--0.4~mag on
week to month timescales) so the LSST sensitivity
and cadence is very well matched to search for 
variability from difference imaging and raw
photometry. The fraction of BCGs exhibiting 
strong, variable AGN will set important
constraints on the duty cycle of activity in 
BCGs and the accretion rate relative to 
the Eddington limit during these AGN outbursts.
These factors are vital to establish how
AGN activity drives the vast amounts of energy
out into the surrounding intracluster gas
through AGN feedback.



\subsection{What else?}


\section{Common themes for discussion with the wider LSST community}

\subsection{Combining LSST and Euclid data products}

\subsection{Combining LSST data with the e-ROSITA cluster catalogues}

\subsection{Requirements on numerical simulations}

\subsection{Requirements on data quality including flat-fielding}

\subsection{Synergies with other Working Groups and Science Collaborations}

\subsubsection{Weak-lensing}

\subsubsection{Strong-lensing}

\subsubsection{Galaxies}

\subsubsection{Others?}

\section{Summary}


\end{document}
