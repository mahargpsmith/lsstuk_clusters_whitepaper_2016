\documentclass[a4paper,11pt]{article}
\usepackage{graphicx,url}
%\usepackage[big,compact,sf]{titlesec}
%\usepackage{sidecap}
\usepackage{graphics,graphicx}
%\usepackage{verbatim}
%\usepackage{wrapfig}
%\usepackage{floatflt}
%\usepackage{times}
%\usepackage{multicol}
\usepackage{color}
\newcommand{\red}{\textcolor{red}}

%\usepackage{sober}
%\usepackage[small,compact,sf]{titlesec}

\newcounter{fred}

\newcommand{\mic}{\mu{\rm m}}
\def\es{\mathrel{\rm e^-/s}}
\def\pix{\mathrel{\rm pix}}
\def\elec{\mathrel{\rm e^-}}
\def\kpc{\mathrel{\rm kpc}}
\def\Gpc{\mathrel{\rm Gpc}}
\def\Msol{\mathrel{\rm M_{\odot}}}
\def\fsub{\mathrel{f_{\rm sub}}}
\def\Mtot{\mathrel{M_{\rm tot}}}
\def\ls{\mathrel{\hbox{\rlap{\hbox{\lower4pt\hbox{$\sim$}}}\hbox{$<$}}}}
\def\gs{\mathrel{\hbox{\rlap{\hbox{\lower4pt\hbox{$\sim$}}}\hbox{$>$}}}}
\def\Msolpyr{\mathrel{\rm M_{\odot}\,yr^{-1}}}
\def\mas{\mathrel{\rm mas}}
\def\pc{\mathrel{\rm pc}}
\def\Ho{\mathrel{H_{\rm 0}}}
\def\oM{\mathrel{\Omega_{\rm M}}}
\def\oL{\mathrel{\Omega_{\rm \Lambda}}}
\def\kms{\mathrel{\rm km\,s^{-1}}}
\def\ms{\mathrel{\rm m\,s^{-1}}}
\def\m{\mathrel{\rm m}}
\def\nm{\mathrel{\rm nm}}
\def\mm{\mathrel{\rm mm}}
\def\cm{\mathrel{\rm cm}}
\def\km{\mathrel{\rm km}}
\def\um{\mathrel{\mu{\rm m}}}
\def\ang{\mathrel{\rm \AA}}
\def\Mpc{\mathrel{\rm Mpc}}
\def\ksec{\mathrel{{\rm ksec}}}
\def\mag{\mathrel{\rm mag}}
\def\Gyr{\mathrel{\rm Gyr}}
\def\Hz{\mathrel{\rm Hz}}
\def\MHz{\mathrel{\rm MHz}}
\def\GHz{\mathrel{\rm GHz}}
\def\THz{\mathrel{\rm THz}}
\def\PHz{\mathrel{\rm PHz}}
\def\EHz{\mathrel{\rm EHz}}
\def\Js{\mathrel{\rm Js}}
\def\J{\mathrel{\rm J}}
\def\W{\mathrel{\rm W}}
\def\eVs{\mathrel{\rm eV\,s}}
\def\eV{\mathrel{\rm eV}}
\def\K{\mathrel{\rm K}}
\def\Jy{\mathrel{\rm Jy}}
\def\mJy{\mathrel{\rm mJy}}
\def\uJy{\mathrel{\rm \mu Jy}}
\def\sr{\mathrel{\rm sr}}
\def\rad{\mathrel{\rm rad}}
\def\deg{\mathrel{\rm deg}}
\def\degsq{\mathrel{\rm deg}^2}
\def\fwhm{\mathrel{\rm FWHM}}
\def\fried{\mathrel{r_0}}
\def\fo{\mathrel{f_{\rm o}}}
\def\fe{\mathrel{f_{\rm e}}}
\def\s{\mathrel{\rm s}}
\def\dol{\mathrel{D_{\rm OL}}}
\def\dos{\mathrel{D_{\rm OS}}}
\def\dls{\mathrel{D_{\rm LS}}}


\newcommand{\captionfonts}{\small}
\makeatletter  % Allow the use of @ in command names
\long\def\@makecaption#1#2{%
  \vskip\abovecaptionskip
  \sbox\@tempboxa{{\captionfonts #1: #2}}%
  \ifdim \wd\@tempboxa >\hsize
    {\captionfonts #1: #2\par}
  \else
    \hbox to\hsize{\hfil\box\@tempboxa\hfil}%
  \fi
  \vskip\belowcaptionskip}
\makeatother   % Cancel the effect of \makeatletter


\setlength{\textwidth}{172mm} 
\setlength{\textheight}{260mm}
\setlength{\topmargin}{-20mm} 
\setlength{\oddsidemargin}{-5mm}
\setlength{\evensidemargin}{10mm} 
\setlength{\headheight}{5mm}
\setlength{\headsep}{5mm} 
\setlength{\hoffset}{0in}
\setlength{\voffset}{0in}

\parskip=2truemm                       % Paragraph spacing
\parindent=4truemm                       % Paragraph indentation

\begin{document}

\pagestyle{myheadings}\markright{LSST:UK Galaxy Clusters White Paper 2016}

\sloppy

\pagestyle{empty}

~\vspace{70mm}

\centerline{\LARGE\bf LSST:UK Galaxy Clusters}
\bigskip\bigskip\bigskip
\centerline{\Large\bf High Level Science Interests and Science Requirements}
\medskip
\centerline{\Large\bf of the UK Galaxy Clusters Community}
\medskip
\centerline{\Large\bf as they relate to LSST}
\bigskip\bigskip\bigskip
\centerline{\Large\bf White Paper 2016}

\vspace{90mm}

\large
\noindent{\bf Last updated}: June 16, 2016

\noindent{\bf Contributors}: Graham P.\ Smith, et al. {\it [add your names here]}


\newpage
\pagestyle{myheadings}
\setlength{\topmargin}{-10mm}
\setlength{\textheight}{255mm}

\tableofcontents

\newpage

\section{Introduction}

{\it [Responsible: Graham Smith]}

\noindent 
The purpose of this document is to summarise and communicate the UK
galaxy cluster community's interests as they relate to future
exploitation of LSST.  Section 2 describes our interests and the high
level requirements on the data.  Section 3 distils the common themes
from our varied interests to form the basis for discussion with other
communities within LSST, both within the UK and our colleagues in the
US.

The initial names attached to science interests in Section 2 is based
on participation/discussion at the inaugural LSST:UK Clusters meeting
on June 7, 2016.  However this document is open for anyone in the UK
to join and contribute to.  The aim is to be succinct, so please stick
to no more than one page per science interest in Section 2.

\section{Science Interests}

\subsection{Intracluster Light and low surface brightness emission}

{\it [Responsible: Chris Collins]}

\subsection{High redshift clusters and protoclusters}

{\it [Responsible: Nina Hatch and Malcolm Bremer]}

\subsection{Numerical simulations}

{\it [Responsible: Ian McCarthy]}

\subsection{Galaxy cluster mass calibration}

{\it [Responsible: Graham Smith]}

\subsection{Machine learning and cluster detection methods}

{\it [Responsible: Jim Geach]}

\subsection{Brightest cluster galaxies}

{\it [Responsible: Alastair Edge]}

\subsection{What else?}

{\it [Responsible: Who else?]}

\section{Common themes for discussion with the wider LSST community}

\subsection{Combining LSST and Euclid data products}

\subsection{Combining LSST data with the e-ROSITA cluster catalogues}

\subsection{Requirements on numerical simulations}

\subsection{Requirements on data quality including flat-fielding}

\subsection{Synergies with other Working Groups and Science Collaborations}

\subsubsection{Weak-lensing}

\subsubsection{Strong-lensing}

\subsubsection{Galaxies}

\subsubsection{Others?}

\section{Summary}


\end{document}
