\documentclass[a4paper,11pt]{article}
\usepackage{graphicx,url}
%\usepackage[big,compact,sf]{titlesec}
%\usepackage{sidecap}
\usepackage{graphics,graphicx}
%\usepackage{verbatim}
%\usepackage{wrapfig}
%\usepackage{floatflt}
%\usepackage{times}
%\usepackage{multicol}
\usepackage{color}
\newcommand{\red}{\textcolor{red}}

%\usepackage{sober}
%\usepackage[small,compact,sf]{titlesec}

\newcounter{fred}

\newcommand{\mic}{\mu{\rm m}}
\def\es{\mathrel{\rm e^-/s}}
\def\pix{\mathrel{\rm pix}}
\def\elec{\mathrel{\rm e^-}}
\def\kpc{\mathrel{\rm kpc}}
\def\Gpc{\mathrel{\rm Gpc}}
\def\Msol{\mathrel{\rm M_{\odot}}}
\def\fsub{\mathrel{f_{\rm sub}}}
\def\Mtot{\mathrel{M_{\rm tot}}}
\def\ls{\mathrel{\hbox{\rlap{\hbox{\lower4pt\hbox{$\sim$}}}\hbox{$<$}}}}
\def\gs{\mathrel{\hbox{\rlap{\hbox{\lower4pt\hbox{$\sim$}}}\hbox{$>$}}}}
\def\Msolpyr{\mathrel{\rm M_{\odot}\,yr^{-1}}}
\def\mas{\mathrel{\rm mas}}
\def\pc{\mathrel{\rm pc}}
\def\Ho{\mathrel{H_{\rm 0}}}
\def\oM{\mathrel{\Omega_{\rm M}}}
\def\oL{\mathrel{\Omega_{\rm \Lambda}}}
\def\kms{\mathrel{\rm km\,s^{-1}}}
\def\ms{\mathrel{\rm m\,s^{-1}}}
\def\m{\mathrel{\rm m}}
\def\nm{\mathrel{\rm nm}}
\def\mm{\mathrel{\rm mm}}
\def\cm{\mathrel{\rm cm}}
\def\km{\mathrel{\rm km}}
\def\um{\mathrel{\mu{\rm m}}}
\def\ang{\mathrel{\rm \AA}}
\def\Mpc{\mathrel{\rm Mpc}}
\def\ksec{\mathrel{{\rm ksec}}}
\def\mag{\mathrel{\rm mag}}
\def\Gyr{\mathrel{\rm Gyr}}
\def\Hz{\mathrel{\rm Hz}}
\def\MHz{\mathrel{\rm MHz}}
\def\GHz{\mathrel{\rm GHz}}
\def\THz{\mathrel{\rm THz}}
\def\PHz{\mathrel{\rm PHz}}
\def\EHz{\mathrel{\rm EHz}}
\def\Js{\mathrel{\rm Js}}
\def\J{\mathrel{\rm J}}
\def\W{\mathrel{\rm W}}
\def\eVs{\mathrel{\rm eV\,s}}
\def\eV{\mathrel{\rm eV}}
\def\K{\mathrel{\rm K}}
\def\Jy{\mathrel{\rm Jy}}
\def\mJy{\mathrel{\rm mJy}}
\def\uJy{\mathrel{\rm \mu Jy}}
\def\sr{\mathrel{\rm sr}}
\def\rad{\mathrel{\rm rad}}
\def\deg{\mathrel{\rm deg}}
\def\degsq{\mathrel{\rm deg}^2}
\def\fwhm{\mathrel{\rm FWHM}}
\def\fried{\mathrel{r_0}}
\def\fo{\mathrel{f_{\rm o}}}
\def\fe{\mathrel{f_{\rm e}}}
\def\s{\mathrel{\rm s}}
\def\dol{\mathrel{D_{\rm OL}}}
\def\dos{\mathrel{D_{\rm OS}}}
\def\dls{\mathrel{D_{\rm LS}}}


\newcommand{\captionfonts}{\small}
\makeatletter  % Allow the use of @ in command names
\long\def\@makecaption#1#2{%
  \vskip\abovecaptionskip
  \sbox\@tempboxa{{\captionfonts #1: #2}}%
  \ifdim \wd\@tempboxa >\hsize
    {\captionfonts #1: #2\par}
  \else
    \hbox to\hsize{\hfil\box\@tempboxa\hfil}%
  \fi
  \vskip\belowcaptionskip}
\makeatother   % Cancel the effect of \makeatletter


\setlength{\textwidth}{172mm} 
\setlength{\textheight}{260mm}
\setlength{\topmargin}{-20mm} 
\setlength{\oddsidemargin}{-5mm}
\setlength{\evensidemargin}{10mm} 
\setlength{\headheight}{5mm}
\setlength{\headsep}{5mm} 
\setlength{\hoffset}{0in}
\setlength{\voffset}{0in}

\parskip=2truemm                       % Paragraph spacing
\parindent=4truemm                       % Paragraph indentation

\begin{document}

\pagestyle{myheadings}\markright{LSST:UK Galaxy Clusters White Paper 2016}

\sloppy

\pagestyle{empty}

~\vspace{70mm}

\centerline{\LARGE\bf LSST:UK Galaxy Clusters}
\bigskip\bigskip\bigskip
\centerline{\Large\bf High Level Science Interests and Science Requirements}
\medskip
\centerline{\Large\bf of the UK Galaxy Clusters Community}
\medskip
\centerline{\Large\bf as they relate to LSST}
\bigskip\bigskip\bigskip
\centerline{\Large\bf White Paper 2016}

\vspace{90mm}

\large
\noindent{\bf Last updated}: June 16, 2016

\noindent{\bf Contributors}: Graham P.\ Smith, et al. {\it [add your names here]}


\newpage
\pagestyle{myheadings}
\setlength{\topmargin}{-10mm}
\setlength{\textheight}{255mm}

\tableofcontents

\newpage

\section{Introduction}

{\it [Responsible: Graham Smith]}

\noindent 
The purpose of this document is to summarise and communicate the UK
galaxy cluster community's interests as they relate to future
exploitation of LSST.  Section 2 describes our interests and the high
level requirements on the data.  Section 3 distils the common themes
from our varied interests to form the basis for discussion with other
communities within LSST, both within the UK and our colleagues in the
US.

The initial names attached to science interests in Section 2 is based
on participation/discussion at the inaugural LSST:UK Clusters meeting
on June 7, 2016.  However this document is open for anyone in the UK
to join and contribute to.  The aim is to be succinct, so please stick
to no more than one page per science interest in Section 2.

\section{Science Interests}

\subsection{Intracluster Light and low surface brightness emission}

{\it [Responsible: Chris Collins]}

\subsection{High redshift clusters and protoclusters}

{\it [Responsible: Nina Hatch and Malcolm Bremer]}


\noindent We will use the LSST survey and deep drilling fields to search for distant ($z >1$) clusters and protoclusters to study the formation of clusters and their member galaxies. Cluster and protocluster detection algorithms will concentrate on searching for galaxy overdensities in redshift space which exhibit strong Balmer/4000\AA break features over large areas ($\sim$ 10\,arcmin$^{-2}$). Based on recent studies (e.g. Chaing et al. 2014), (proto)cluster detection will require photometric redshifts with at least $\Delta z/ (1+z)\sim2.5$\% precision at $z>1$. This requires a strong synergy between Euclid and LSST data as the Euclid Y, J and H images are essential to span over the Balmer/4000\AA\ break for $z>1.5$ galaxies.  

The depth of the final LSST catalogues are sufficient to detect significant numbers of red galaxies within each (proto)cluster at redshifts up to $z\sim1.7$. Clusters and protoclusters up to $z\sim2.5$ can be detected in the deep drilling fields using similar algorithms, and these fields  will produce cleaner cluster samples at $z<1.7$ due to their lower photometric redshift uncertainty.

For protoclusters at $z>2.5$, the Balmer/$4000$\AA\ break shifts beyond the wavelength range of LSST and Euclid. Furthermore, the red sequences of such distant protoclusters are less prominent (perhaps non-existent). We therefore must use the Lyman break feature of galaxies to locate $z>2.5$ protoclusters.

To locate protoclusters the images must be deep enough to robustly identify a galaxy overdensity across a few sq. arcmin. This requires us to detect at least $\sim2$ LBGs per sq. arcmin. To achieve this the filter redward of the Lyman break must reach depths of 26-27 mag (AB), and the filter blueward of the break must reach 27-28 mag [but more detailed simulations are needed]. This means:\\
\noindent $\bullet$~$z\sim3$ U-band dropouts will only be detected to sufficient depths in the deep drilling fields as the U-band in the main field survey is too shallow. \\
\noindent $\bullet$ ~Dropouts in the g- and r-bands  ($3.5<z<4.5$) will be detected to sufficient depths to locate protoclusters in the main survey. \\
\noindent $\bullet$ ~The surface density of i- and z-band dropouts will be too low to robustly identify protoclusters in both main and deep drilling fields. [Although this is subject to change as the depths of the deep fields are not certain at this stage]

In summary, combining the LSST and Euclid surveys allows us to search for $1<z<1.7$ clusters and $3.5<z<4.5$ protoclusters across the main survey using photometric redshifts and Lyman break dropout techniques. The deep drilling fields allow us to search for $z>1.7$ (proto)clusters using photometric redshifts, and $z\sim3$ protoclusters using U-band dropouts. The main requirement for locating distant clusters and protoclusters is a good synergy between Euclid and LSST data products. We require accurate photometric catalogues based on Euclid detection images. The deep drilling fields are very important and infrared (3.6 and 4.5\,$\mu$m) coverage of these fields is highly desirable.

In preparation for LSST data we should develop and test algorithms to locate distant clusters and protoclusters. Realistic simulated light cones would help test these algorithms. Such light comes need to be sufficiently large to contain significant numbers of high redshift clusters and protoclusters, and they would be most useful if the luminosity function of galaxies (at LSST and Euclid wavelengths) matched that of the local Universe and up to $z\sim4$.

\subsection{Numerical simulations}

{\it [Responsible: Ian McCarthy]}

\subsection{Galaxy cluster mass calibration}

{\it [Responsible: Graham Smith]}

\subsection{Machine learning and cluster detection methods}

{\it [Responsible: Jim Geach]}

\subsection{Brightest Cluster Galaxies}

{\it [Responsible: Alastair Edge]}

The Brightest Cluster Galaxy (BCG) in a cluster is in most
clusters substantially brighter than all other cluster
members (by 1--1.5~mag) and found at the dynamical centre
of the cluster at, or close to, the X-ray peak (Sanderson
et al. 2009). These galaxies represent the most massive 
stellar systems known and their properties can be used
to constrain their formation history and how it relates
to the evolution of their host cluster.

Most BCGs are dominated by an old stellar population
and share the same colour-magnitude sequence with the
other early type galaxies in the cluster. However, there
is a significant population of BCGs that exhibit recent
star formation and/or an AGN core (Crawford et al. 1999).
This subset of ``active'' BCGs are predominately found in
the most X-ray luminous clusters (Green et al 2016) and
have associated optical line emission and cold molecular
gas (Edge 2001). The properties of the intracluster gas 
surrounding the BCG appear to be the dominant factor
in triggering this activity (Cavagnolo et al 2008) with
a sharp threshold in central entropy marking the 
point where $>$90\% of BCGs below show optical line emission 
and $<$5\% of those above have lines. 

Therefore, if it is possible to identify which BCGs are
significantly bluer expected from the observed colours
of the cluster population, we can very efficiently
photometrically select these objects and compare
their radio, MIR and X-ray properties to the 
more passive BCGs. In the radio the average power
of a BCG which exhibits optical line emission is
an order of magnitude higher than that in non-line
emitting systems (Hogan et al. 2015). 

The depth and coverage of LSST, when combined
with NIR and MIR data from Vista VHS and AllWISE,
will be able to select these peculiar BCGs in
any cluster within $z\sim 1.0$ and the more
extreme examples out to $z>1.5$. The combination
of multi-band photometry and multi-frequency
radio data will allow us to determine if the 
AGN and star-formation activity in BCGs evolves
significantly over the majority of the lifetimes
of all clusters.

In addition to the photometric properties
of BCGs, LSST will also provide important
morphological information. For instance,
N-body simulations suggest that the majority
of the growth of BCGs is through the merger
with other massive, early type galaxies 
(De Lucia \& Bliazot 2007). Therefore a
significant fraction of BCGs should have 
the multiple nuclei expected for two 
galaxies in the latter stages of these
massive mergers. These systems are known
but a significant fraction may be chance
alignments given the very high surface density of galaxies
in a cluster core. The quality and coverage of
LSST will provide the statistics to assess
the probabilty of chance alignment and the
relative brightness of the components
in a BCG to the likely merger rate of BCGs
can be determined as a function of redshift
and cluster mass.

The LSST imaging will also identify BCGs
with extended stellar envelopes,
usually refered to as cD haloes,
that extend on $\approx$100~kpc scales
blending into what can be defined
as Intracluster Light (see Chris's section?).
This extended halo is often quite asymmetric
and can contain shells and other evidence
of merger activity. Again, the depth and
consistency of the LSST data will allow us
to determine the properties of these 
extended haloes with other cluster properties
over a wide range in redshift.

Finally, a small fraction of BCGs contain a
sufficiently bright active nucleus that
variability can be detected. The most prominent
example of this is NGC1275 in the Perseus cluster
(Kingman \& O'Connell et al. 1979) which has shown 
two episodes of strong AGN activity in the past
50 years (Dutson et al. 2014). The optical
variability is significant (0.2--0.4~mag on
week to month timescales) so the LSST sensitivity
and cadence is very well matched to search for 
variability from difference imaging and raw
photometry. The fraction of BCGs exhibiting 
strong, variable AGN will set important
constraints on the duty cycle of activity in 
BCGs and the accretion rate relative to 
the Eddington limit during these AGN outbursts.
These factors are vital to establish how
AGN activity drives the vast amounts of energy
out into the surrounding intracluster gas
through AGN feedback.



\subsection{What else?}

{\it [Responsible: Who else?]}

\section{Common themes for discussion with the wider LSST community}

\subsection{Combining LSST and Euclid data products}

\subsection{Combining LSST data with the e-ROSITA cluster catalogues}

\subsection{Requirements on numerical simulations}

\subsection{Requirements on data quality including flat-fielding}

\subsection{Synergies with other Working Groups and Science Collaborations}

\subsubsection{Weak-lensing}

\subsubsection{Strong-lensing}

\subsubsection{Galaxies}

\subsubsection{Others?}

\section{Summary}


\end{document}
